% Chapter 1

\chapter{Theory} % Main chapter title

\label{Theory} % For referencing the chapter elsewhere, use \ref{Chapter1} 

%----------------------------------------------------------------------------------------



%----------------------------------------------------------------------------------------
\section{Cultural differences in Perception}
Cultural differences affect more than just how consumers behave, it also affects how users perceive information. According to (bla and bla) "good quote" \cite{Holistic_vs_Analytic}
\section{User Centred design}
\section{Usability}
\section{User Experience}
User Experience (UX) is an expression popularised by Donald Norman and Jakob Nielsen in "The design of everyday things" (länka på rätt sätt)(Källa!). Norman and Nielsen define User Experience in the following way “True user experience goes far beyond giving customers what they say they want, or providing checklist features. In order to achieve high-quality user experience in a company’s offerings there must be a seamless merging of the services of multiple disciplines, including engineering, marketing, graphical and industrial design, and interface design” (Norman and Nielsen, 2016). (latex quota på rätt sätt). 
\\\\
The term "User Experience" is a widely applied term in web design and it is associated with a multitude of definitions. The International Organization for Standardization (ISO) has also attempted to define the term. Specifically, the organization, in ISO 9241-210, defines UX as “A person's perceptions and responses that result from the use and/or anticipated use of a product, system or service.” Further, the ISO states that “UX includes all users' emotions, beliefs, preferences, perceptions, physical and psychological responses, behaviours and accomplishments that occur before, during and after use”. The standard also states that “UX is a consequence of functionality, system performance, interactive behaviour and assistive capabilities of the interactive system, the user's internal and physical state resulting from prior brand image, presentation, experiences, attitudes, skills and personality, and the context of use” 


\section{Elements of Web Design}
\section{F-shaped Pattern}
The F-shaped pattern refers to findings made in a study on user eye movements  \cite{pernice2014people} (find correct article for f-shaped pattern and cite it here as well). This pattern has been dubbed the "F-shaped pattern" since the study found that users often scan through pages starting with a horizontal movement, usually across the upper part of the content area. Users then tend to read across in a second horizontal movement further down on the page that typically spans a shorter area. Lastly, users scan the content’s left side in a vertical movement. When measuring users' eye gazing as a heat map, this creates a pattern that resembles an F. Web developers, either knowingly or unknowingly, often design their websites according to this pattern. The F-shaped pattern is not an absolute law and several other scanning patterns exist, but the F-shaped pattern remains the most prevalent in western cultures. \cite{f-shape_today} If a developer designs a page without knowledge about this pattern, they risk putting important information in places where users might miss it. The F-shaped pattern is mostly prevalent in western cultures, where the studies have been conducted.

\section{Perception in asia (f-shaped pattern.)}

\section{Natural Mapping}
\section{User Testing}

\section{Usability Metrics}
There are several different types of metrics that can be used to measure the usability of a prototype or product. Among these metrics are performance metrics, Issues-based metrics, self-reported metrics, behavioral metrics, comparative metrics, and etc \cite{tullis_albert_2011}. For this project, we have decided to focus on performance metrics and self-reported metrics. Usability metrics are powerful tools that are generally under utilized by most companies \cite{norman_metrics}. 
\subsection{Performance Metrics}
Performance metrics can be used to measure a user's behavior when interacting with a product. In this project, the performance metric data will automatically be gathered and sent to a database. This data will then be further analyzed to gain a deeper understanding of how the users engage with the product. (DOUBLE CHECK ->)To be statistically significant the data gathered with a appropriate confidentiality interval at least eight participants are needed \cite{tullis_albert_2011}.(<- DOUBLE CHECK)

 There are 5 basic performance metrics which include: \cite{tullis_albert_2011} \begin{itemize}
\item Task success
\item Time-on-task
\item Errors
\item Efficiency
\item Learnability
\end{itemize}
In order to measure the task rate success metric, the required task must be clearly defined and have an unambiguous goal. For example, "send an email to person x" reflects a well structured task, thus, task success rate can be accurately measured given the well-defined objective. Conversely, the task, "research budget car brands" does not offer clear guidelines and can be responded to in multitude of ways due to its ill-defined objective and, as such, would not be suitable for measuring task success. There are two types of task success forms. The first is represented as a binary measure; either the user completes or fails to complete the task \cite{tullis_albert_2011}. The second type is continuous rather than categorical as it measures the level of success. This metric is particularly useful if the given task can be partially completed. One example of a task that would benefit from partial success rate measurements would be if we ask users to open a specific video on YouTube, but the user opens the wrong video. The user would still be somewhat correct as she correctly navigated to YouTube, but failed to select the correct video, yielding a partial success. The simplest way of measuring success levels is to assign numeric values to the study subject's performance (i.e., this might range from 1 to 10, where 5 represents that the user completed half of the task).

\\
There are several ways in which users' can fail to accomplish their tasks. Users may, for instance, incorrectly assume the task has been completed when, in fact, it is only partially complete. Further, users may give up on trying to solve the task out of frustration or users may falsely believe they successfully finished the task when, in actuality, they performed the wrong task. This data can be invaluable to the process of uncovering how well users understand a given system. 
\\\\
Time-on-task is a simple measure; it simply logs the time users' spent to complete or fail the task at hand.
\\\\
Errors in this case do not refer to programmatic errors, but to mistakes made by the user. One example of an error is when users select an incorrect tab before finding the correct one. In this example, every wrong path and/or click used in performing the task, except the optimal one, is logged as an error. Error measurements can assist developers in gauging how well a user understands the website as well as how intuitive the website is for a first time user. 
\\\\
Efficiency is analogous to the Time-on-task measure, but it can also be measured by how many steps users took to complete the task. It is important to note that efficiency should only be measured on successful tasks \cite{tullis_albert_2011}.
\\\\
Learnability represents the degree to which a user becomes more efficient at using a product over time. Essentially, the amount of time reduced in completing a task from the first iteration to the second or third provides an indication of how well the user learned to interact with the product. 
\subsection{Self-Reported Metrics}
Self-reported metrics are employed when directly asking users to describe their experience engaging with the product. One means of obtaining self-reports is through surveys. A common method for doing this is by employing the System Usability Scale ()SUS) created by John Brooke \cite{tullis_albert_2011} \cite{brooke1996sus}. SUS is a survey containing 10 questions with a scale from 1-5, where 5 corresponds to "strongly agree" and 1 reflects "strongly disagree", for each question (See appendix X). SUS is a metric tool, used over 22 years, that has been proven to be a streamlined and robust tool for gauging usability \cite{brooke1996sus}.  (SEE APENDIX for SUS ecample )
\section{Usability Testing}

\section{Colour and Culture}
Different cultures have always preferred distinct colour schemes, and this variation has also effected the degree to which a user trusts or likes a website. Not all people prefer the same colour layouts and a study conducted by (XXXX) \cite{Color} indicates that this colour bias may be rooted in culture. The study suggests that the colour schema a website uses influences the trust and likeability of the web page. The study further indicates that individuals hailing from different cultural backgrounds tend to select colours associated with their culture. These cultural factors must be accounted for when designing websites for any given culture as colour schema impacts how well users like and engage with the website. Selecting colours that users from a certain culture feel more comfortable with can be vital to enhancing the user's experience with the site.
\section{Trends}
Trends, simply something that is popular in a particular moment, are an ever-present phenomenon. However, simply because a certain behaviour or style is trending does not necessarily indicate that it is the most optimal way to perform an action; on the contrary, usually the opposite is true. Comparing design trends to actual usability in this research means that we will examine if there is any actual underlying data that supports the trend from a usability perspective. This can have two outcomes: either the trend has grown because it more closely caters to how users use the respective products or the trend is a by-product from how designs were previously created. One example of this is that we load more information than necessary on to a page because we have previously done so. The reason we started doing this was due to slow internet speeds which lead to large loading times when clicking through a page. As such, now, even if the internet speed is quick and we do not have to load all information to a page we still do it since both users and developers have become accustomed to this pattern.

\section{Culture and Usability}

\section{Great Firewall of China}
The Great Firewall of China (GFC) is a combination of laws and technologies the Chinese government uses to domestically regulate the internet. Examples of services blocked by the GFC are Google, Facebook, and Youtube. The GFC also artificially causes traffic from abroad to be significantly slower than applications hosted in China. Hosting an application on a server in China requires a specific IPC license from the Chinese government and getting one is a long and slow bureaucratic process. The sort of algorithms that are used by GFC are largely unknown and can be challenging to circumvent.
\section{Asynchronous}
Asynchronous programming refers to the task of making several data processors run in parallel to each other usually without impacting one another. Asynchronous parallel processes are often called threads. One example of this would be one thread working on reacting to a users request and supplying that user with the correct information. Simultaneously, another thread, not visible to the user, is saving all the users actions and sending them to a server. (find a source)  

\section{AWS - Amazon Web Services}
Amazon Web Services (AWS) is the world's largest provider of web-hosting. Amazon allows their customers to easily host applications globally and the company provides several features to assist their customers with this task. (cite to amazon here and all bellow)
\subsection{EC2}
EC2 (Elastic Cloud Compute) is a basic web server service that AWS offers. EC2 allows customers to set up a virtual server with different amounts of CPU, Memory  and etc.These servers can be set up on several AWS locations across the world. This server can be customised to run an operating system of the customer's choice, the most common of which are Linux and Windows.
\subsection{Auto scaling}
Auto scaling is a feature provided by AWS that automatically scales up the server in case of increased traffic. This mean if a application has a large of amount of traffic on a server the auto scaling functionality create an extra server  that can handle user requests. Auto scaling also allows for automatic scale down in case of low traffic. 
\subsection{Load balancing}
Load balancing is a feature from AWS that automatically balances the load of the EC2 instances. If a user has three EC2 instances, the load balancing will make sure that the workload is shared by all EC2 instances. This helps to prevent one instance from overloading.
\subsection{RDS}
Relational database service (RDS) is a database service provided by AWS. RDS lets customers set up a database of their choice and host it on AWS servers. Customers can set this database up on several locations across the world and configure it to suit their application.
\subsection{S3}
S3 is an AWS feature that allows for object storage in the cloud. S3 allows the user to store anything he deems fit - this be everything from files and Images to code repositories. Images that are used on websites can stored on S3 and then downloaded to the website when the user opens it, this is a common way to handle images in web sites and applications.
\subsection{Elastic Beanstalk}
Elastic Beanstalk also called EB is a feature provided by AWS that automatically sets up a instance complete environment with auto scaling, load balancing, Relational database and EC2 instances.
\section{React-Redux}
React is a front-end JavaScript library developed by Facebook. React is based on the user building and reusing components. This allows for very structured and highly scalable code.
\\\\
Handling data-flow in a react application can be very tricky, this is where redux comes in. Redux is a JavaScript library that allows for structuring and handling of a web application's data flow in a structured way. React and Redux are so commonly used together that libraries combining them have been made. React-Redux is the most popular use of these libraries and they work very well together to allow scalable and reusable code. 
\subsection{Redux-saga}
referense(https://redux-saga.js.org/docs/introduction/ and https://github.com/redux-saga/redux-saga) Redux-saga is a javascript library that is made to handle a applications asynchronous tasks. Redux-sagas is often used for data feching and posting. It can also be used for other asyncronius tasks. Sagas handle asynchronous tasks without the user getting impacted at all by what goes on in the background.

\section{MySql database}
MySql is a version of the database query language SQL. SQL has been used since 1981 and is used to set-up, save and get information from a database. MySql is free to use and has a public license. Mysql is a language that is both simple to use and quite powerful. Setting up inputting and getting data from an SQL database can be done through only a  few lines of code.

\section{API}
Api (application programming interface) is a interface between the front-end and server. A api allows the application to communicate with functions and servers outside the internal environment. Examples of these are databases, other servers and other api's. Ans api allows for clearer communication between different actors on the web. 

\section{Statistical significance}

\section{T-plot (whatever it's called)}
\section{Null hypothesis}
\section{P value}
\section{waterfall}



