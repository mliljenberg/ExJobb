% Chapter 1

\chapter{Theory} % Main chapter title

\label{Chapter1} % For referencing the chapter elsewhere, use \ref{Chapter1} 

%----------------------------------------------------------------------------------------



%----------------------------------------------------------------------------------------
\section{Cultural differences in Perception}
Cultural differences affect more than just how we behave it also can affect how we perceive information. According to (bla and bla) "good quote" \cite{Holistic_vs_Analytic}
\section{User Centred design}
\section{Usability}
\section{User Experience}
\section{Elements of Web Design}
\section{F-shaped Pattern}
The F-shaped pattern regards to a finding made in the xxx study \cite{pernice2014people} (find correct article for f-shaped pattern and cite it here as well). This pattern is named the "F-shaped pattern" since the study found that users often scan through the page starting with a horizontal movement, usually across the upper part of the content area. Then the users read across in a second horizontal movement further down on the page that typically covers a shorter area. Lastly users scan the content’s left side in a vertical movement. When measuring the users eye gazing as a heat map this creates a pattern that resembles a F. Quite a few web pages either knowingly or unknowingly have designed their websites in regards to this pattern. The F-shaped pattern is not a absolute law and there exists several other scanning patterns but the F-shaped pattern is still the most prevalent in western cultures. \cite{f-shape_today} If a website design a page without knowledge about this pattern they run the risk of putting important information in places where their users might miss it. The F-shaped pattern is mostly prevalent in western cultures where the studies have been conducted.

\section{Perception in asia (f-shaped pattern.)}

\section{User Testing}
\section{Natural Mapping}
\section{Usability Metrics}
There are several different types of metrics that can be used to measure the usability of your prototype/product. Among them there are performance metrics, Issues-Based Metrics, Self-Reported Metrics Behavioral Metrics, Comparative metrics etc \cite{tullis_albert_2011}. For this project we have chosen to focus on Performance Metrics and Self-Reported Metrics. Usability metrics is a very powerful tool that is usually under utilized by most companies \cite{norman_metrics}. 
\subsection{Performance Metrics}
Performance Metrics can be used to measure the users behavior when using a product. In this project the performance metric data will be automatically gathered. This data can then later be analyzed to gain a greater understanding for the users. To be statistically significant the data gathered with a appropriate confidentiality interval at least eight participants are needed \cite{tullis_albert_2011}. There are 5 basic performance metrics which include: \cite{tullis_albert_2011} \begin{itemize}
\item Task success
\item Time-on-task
\item Errors
\item Efficiency
\item Learnability
\end{itemize}

To be able to measure the task success metric the task at hand has to be clearly defined and have a clear end. "Send a email to x" is a good task were task success can be successfully measured. The task "research cheap car brands" on the other hand does not have a clear end defined and is therefore not suitable for measuring task success. \\ 
There are two different types of task Success. The first is a binary measure either the user is able to complete the task or not \cite{tullis_albert_2011}. The second type is to measure the level of success. This is a useful measure if the task can be partly completed, one example of a task that could be measured with the help of partial success would be ..... The simplest way to measure level of success is to assign it a numeric value. A example of this might be from 0-1 where 0.5 means the user has halfway completed the task.
\\
There is several ways a user can fail in a task. The user may think the task is completed when in fact it is only partially completed, the user might give up on trying to solve the task or the user might completely think he has successfully finished the task while he might not have done the correct task at all. This data can be very useful and will be able to a higher degree tell you how well a user understands the system.
\\\\
Time-on-task is a very simple measure it simply tells you the time it took the user to complete or fail the task at hand.
\\\\
Errors in this case is not referred to programmatic errors but mistakes made by the user. One example of a error could be a goes in to a wrong tab before finding the correct one. In this example every wrong path/click to be able to perform the task except the optimal one is a error. Error measurements can help us how well the user is understanding the website and how intuitive the website is for a first time user. 
\\\\
Efficiency can be seen as the same as Time-on-task measure, but it can also be measured by how many steps the user had to take to complete the task. It is important to note that efficiency should only be measured on successful tasks \cite{tullis_albert_2011}.
\\\\
Learnability can be seen as to how high degree does the user become more efficient at using the product over time. Basically the time reduction of completing the task the second or third time will tell us how well the user learned to use the product.  
\subsection{Self-Reported Metrics}
Self-Reported Metrics ask the user what he thought of the product. A way to do this is by using a form. A common method for doing this is by using System Usability Scale also called SUS \cite{tullis_albert_2011} \cite{brooke1996sus}. Sus is a method created by John Brooke. SUS is a form containing ten questions with a scale from 1-5 where 5 is "Strongly agree" and 1 is "Strongly disagree". See (appedix x) for a example of the form. SUS is a metric tool that have been used and proven over 22 years to be a robust and simple tool for measuring usability \cite{brooke1996sus}.  (SEE APENDIX for SUS ecample )
\section{Usability Testing}
\section{Colour and Culture}
Different cultures have always had a focus on different colours, this has also have a effect to what degree a user trust and like a website. Not all people prefer the same colour scheme and study made by (XXXX) \cite{Color} shows that this colour preference can also be cultural. The study showed that the colour schema a website use affect the trust and how well liked a website can be. It also showed that people from different cultures have a preferences for colours associated with that culture. This is something that has to be taken into account when designing a website for an certain culture this since the correct colour schema can affect how well the users will like and interact with the website. Using colours that the users from a culture feel more comfortable with can be very important to enhance the users experience when using the site.
\section{Trends}
\section{Culture and Usability}

