% Chapter Template

\chapter{Phase 2 - Prototyping} % Main chapter title

\label{Chapter5} % Change X to a consecutive number; for referencing this chapter elsewhere, use \ref{ChapterX}

%----------------------------------------------------------------------------------------
%	SECTION 1
%----------------------------------------------------------------------------------------
The goal of Phase 2 is to quickly create prototypes for our design that achieve what we want for the project. The prototypes will then be tested so that the actual website will have some testing behind it before building and thereby enable a quicker development.

\section{Method}
Test method for low-fi and high-fi prototype. The Low-fi prototype was a simple skiss made on paper. The high-fi prototype was made in a program called sketch.

\section{Results}
\subsection{Low-fi Prototype}
The Low-fi prototype was quickly sketched with pen on paper. Since one of the websites was almost a direct imitation of two current large Chinese and English news sites those where only quickly showed for people with different ethnicity to see that everything looked correct and nothing was missed. The main focus with the Low-fi prototype was spend on the second site that was not made directly from any external source. Firstly a quick paper prototype was drawn on paper  (see figures...). These figures where then showed from a ux-design specialist in China and Sweden. A new model was drawn according to feedback and showed/tested on some potential users from China and on some from Sweden. (Write how the test was conducted with test methodology etc....) This feedback was then used to create a High-fi Prototype.

\subsection{High-fi Prototype}

\subsection{Testing of High-fi Prototype}

\section{Conclusion Phase 2}


\subsection{Limitations}