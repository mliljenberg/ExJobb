% Chapter Template

\chapter{Phase 2 - Prototyping} % Main chapter title

\label{Chapter5} % Change X to a consecutive number; for referencing this chapter elsewhere, use \ref{ChapterX}

%----------------------------------------------------------------------------------------
%	SECTION 1
%----------------------------------------------------------------------------------------
The goal of Phase 2 is to quickly create prototypes for our design that can be used in a pilot test to ensure that the real test will work as intended before it is created. 

\section{Method}
The Low-fi prototype was a simple sketch made with paper and pen. Since the prototype are modelled after an existing website this prototype is focused on the parts needed for testing the page and not the design of the web pages themselves. The low-fi prototype was then discussed with a UX-expert and improved upon.
\\\\
Two High-fi prototypes was made from the online news site BBC \cite{bbc} and QQ \cite{qq_homepage}. These prototypes were directly modelled from the websites and then the corresponding logos were removed. Both the high-fi prototypes are then translated to English respectively Chinese using Google translate. This translated text was then looked over by the Tetra pak supervisor who is a native Chinese speaker to ensure that the translation was correct. Test questions and tasks was created for both the BBC and the QQ site in English. These questions were also translated to Chinese for the Chinese users.
\\\\
A questionnaire modelled after the Sus (system usability scale) was then created. The purpose of this questionnaire is to gage what the user feel about using the websites. Many of the original questions did not suit this purpose and was therefore either removed or changed.

\section{Results}
\subsection{Low-fi Prototype}
The Low-fi Prototype can be seen in the following figure (fig:.....). As can bee seen in the images only the testing part and not the design of the website was created. The improvements from this prototype can be seen in the high-fi prototype.

\subsection{High-fi Prototype}
The High-fi prototypes can be seen in the following figures: QQ (CITE to QQ image), BBC (SITE TO BBC IMAGE). The test parts can be seen in the following image(ref to that figure)

\subsection{Sus}
The Sus inspired questionnaire contained the following questions:
\begin{enumerate}
	\item I liked the design of the site.
	\item The design of this site was similar to other news sites.
	\item I think that I would like to use this site frequently.
	\item I found the site unnecessarily complex.
	\item I thought the site was easy to use.
	\item I found the various functions in this site were well integrated.
	\item I thought there was too much inconsistency in this site.
	\item I would imagine that most people would learn to use this site very quickly.
	\item I found the site very cumbersome to use.
	\item I felt very confident using the site.
	\item I thought that the amount of information on this site was.
\end{enumerate}
For each of these questions the user could give it a rating from 1 "Strongly disagree" to 5 "Strongly agree". The last question the user instead got the choices 1 "Too sparse" to 5 "Too much".

\section{Discussion}
The main purpose of creating and testing these high-fi prototypes was to see how well the site would work when translating to another language. The site got a very different look after being translated, partly because the Chinese language produces smaller sentences. The Chinese characters also give a very different impression since the design of the characters and general lack of fonts when using. The Chinese QQ site became quite a bit longer with the translated English text. And many sentences had to be spread out over two rows. For the translations made from Chinese to English, many news had to be slightly changed to make logical sense. The translator handled English to Chinese quite a lot better and only small changes needed to be done. These changes in between version should not have any effect on the testing results since what is being tested are how the users handle information density and placement not the actual content of the news.
\\\\
 The questions/tasks for the users to perform on the followed a simple pattern. Roughly half of the information the user was asked to find was located inside the F-shaped pattern and the other half outside the pattern. Some questions were selected to test the different menu-bars as well. The questions that were created can be seen in the pilot study (Ref pilot study).


\section{Conclusion}
To make sure the test will work and to see if we will get any sort of interesting result from the real test before programming the websites a pilot study using the high-fi prototype was conducted \ref{Pilotstudy}).



