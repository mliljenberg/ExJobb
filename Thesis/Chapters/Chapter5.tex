% Chapter Template

\chapter{Phase 2 - Prototyping} % Main chapter title

\label{Chapter5} % Change X to a consecutive number; for referencing this chapter elsewhere, use \ref{ChapterX}

%----------------------------------------------------------------------------------------
%	SECTION 1
%----------------------------------------------------------------------------------------
The goal of Phase 2 is to quickly create prototypes for our design. The prototypes will then be tested and further developed. A pilot study using high-fi prototype will then be preformed to check how well the the study will work when performed online.

\section{Method}
Test method for low-fi and high-fi prototype. The Low-fi prototype was a simple sketch made on paper. The high-fi prototype was made in a program called sketch. A questionnaire was designed according to a modified system usability scale and where also tested in the pilot study. 


\section{Results}
\subsection{Low-fi Prototype}
The Low-fi prototype was quickly sketched with pen on paper. The low-fi prototype was mostly made for planning purposes. The sites used for the test already existed so checking what functions and part of the current websites with a low-fi prototype would have no purpose four our test.

Parts outside of the defined web-pages
The parts of the test that exists outside of the already defined news site low-fi prototypes were created and tester on. This since these parts needed to be defined for this test and had to be made so they would not interrupt the flow of the original website. A image of a low-fi prototype that was made for this part can be seen below (ADD IMAGE). This low-fi was tested iteratively before creating a high-fi prototype. 


\subsection{High-fi Prototype}
Two High-fi prototypes was made from the online news site bbc \cite{bbc} and QQ \cite{qq_homepage}. These prototypes was directly modelled from the websites and then the corresponding logos was removed. The websites was also both translated to English respectivly Chinese. The High-fi prototypes can be seen in the following figures: QQ (CITE to QQ image), BBC (SITE TO BBC IMAGE). The main purpose of creating and testing these high-fi prototype was to see how well the site would work when translate to another language. The site got a very different look after being translated partly because the Chinese language produce a lot smaller sentences. To translate from English to Chinese and vice-versa Google translate was used. This translated text was then looked over by the Tetra pak supervisor who is a native Chinese speaker. 
\\\\
The second benefit of creating and testing the high-fi prototype was to determine what questions worked well for the test and were understood in both English and Chinese. The questions were first created in English and then translated to Chinese. The questions selected was roughly half inside the F-shaped pattern (cite somehting maby?) and half outside the pattern. Some questions were selected to test the different menu-bars as well. The questions selected can be seen in the pilot study (cite pilot study).


\section{Discussion Phase 2}
Both of the pages were modelled from a combination of big websites that are already existing. Because of this the low-fi prototype had very limited benefit to test since what we wanted test already exists. The high-fi prototype on the other hand needed to be tested together with the questions to make sure these were understandable. The pilot study was also conducted to make sure that the main test is feasible. 

\section{Conclusion}
To make sure the test will work and to see if we will get any sort of interesting result from the real test before programming the websites a pilot study using the high-fi prototype was conducted \ref{Pilotstudy}).



