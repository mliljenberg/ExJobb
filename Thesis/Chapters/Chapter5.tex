% Chapter Template

\chapter{Phase 2 - Prototyping} % Main chapter title

\label{Chapter5} % Change X to a consecutive number; for referencing this chapter elsewhere, use \ref{ChapterX}

%----------------------------------------------------------------------------------------
%	SECTION 1
%----------------------------------------------------------------------------------------
The goal of Phase 2 is to quickly create prototypes for our design that achieve what we want for the project. The prototypes will then be tested so that the actual website will have some testing behind it before building and thereby enable a quicker development.

\section{Method}
Test method for low-fi and high-fi prototype. The Low-fi prototype was a simple skiss made on paper. The high-fi prototype was made in a program called sketch.
\subsection{Pilot study}
The pilot study was done by showing the test people the developed sketch prototype. Using this sketch prototype i sat next to the user and showed them what i asked them to do written down on a piece of paper (In chinese for the chinese people and in english for the western). First the people got a minute to look around the page to get a quick feel for the layout of the page. Then a question was showed to the user and a timer was started at the same time. When the test subject found the requested image or text they indicated that they had found the information and the timer was then stopped. This was repeated until all the tasks where fulfilled.

\section{Results}
\subsection{Low-fi Prototype}
The Low-fi prototype was quickly sketched with pen on paper. Since one of the websites was almost a direct imitation of two current large Chinese and English news sites those where only quickly showed for people with different ethnicity to see that everything looked correct and nothing was missed. The main focus with the Low-fi prototype was spent on the second site that was not made directly from any external source. Firstly a quick paper prototype was drawn on paper  (see figures...). These figures where then showed from a ux-design specialist in China and Sweden. A new model was drawn according to feedback and showed/tested on some potential users from China and on some from Sweden. (Write how the test was conducted with test methodology etc....) This feedback was then used to create a High-fi Prototype.

\subsection{High-fi Prototype}
\subsubsection{News Site}
Two High-fi prototypes was made from the online news site bbc \cite{bbc} and QQ \cite{qq_homepage}. These prototypes was directly modelled from the websites and then the corresponding logos was removed. The websites was also both translated to English respectivly Chinese. The High-fi prototypes can be seen in the following figures: QQ (CITE to QQ image), BBC (SITE TO BBC IMAGE)

\subsection{Pilot study}
The BBC pilot study resulted in the following results: (Table with results BBC)

The QQ pilot study resulted in the following: (Table with results QQ)



\section{Conclusion Phase 2}
\subsection{Pilot study}

Pilot studie/ux studie med kineser:
the question angry sport coach did not seem to work well. I think it was badly formulated becuse people did not seem to understand what to look for. Some of the news did not test the F-shaped pattern correctly since they could be found on more than one place on the page (koeran and samsung heir questions). I have not looked thorugh the data yet but it seems like everyone might have about the same search pattern when looking for something specific. These people might also have been influensed by the western culture and this can affect the results.....

QQ: maby i should remove to easy questions regarding images, it seems like this does not test information density on qq since there actually are not that many images on the site.

The questions where also showed in a sequence that might affect the results... Many times the content was close to each other which led to some subjects finding the information quicker beacuse of this (maby this should be included in a more tested way... to actually test it).


\subsection{Limitations}