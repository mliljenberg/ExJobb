% Chapter Template

\chapter{Phase 2 - Prototyping} % Main chapter title

\label{Chapter5} % Change X to a consecutive number; for referencing this chapter elsewhere, use \ref{ChapterX}

%----------------------------------------------------------------------------------------
%	SECTION 1
%----------------------------------------------------------------------------------------
The goal of Phase 2 is to quickly create prototypes for our design that achieve what we want for the project. The prototypes will then be tested so that the actual website will have some testing behind it before building and thereby enable a quicker development.

\section{Method}
Test method for low-fi and high-fi prototype. The Low-fi prototype was a simple sketch made on paper. The high-fi prototype was made in a program called sketch. A questionnaire was designed according to a modified system usability scale and where also tested in the pilot study. 


\section{Results}
\subsection{Low-fi Prototype}
The Low-fi prototype was quickly sketched with pen on paper. Since one of the websites was almost a direct imitation of two current large Chinese and English news sites those where only quickly showed for people with different ethnicity to see that everything looked correct and nothing was missed. The main focus with the Low-fi prototype was spent on the second site that was not made directly from any external source. Firstly a quick paper prototype was drawn on paper  (see figures...). These figures where then showed from a ux-design specialist in China and Sweden. A new model was drawn according to feedback and showed/tested on some potential users from China and on some from Sweden. (Write how the test was conducted with test methodology etc....) This feedback was then used to create a High-fi Prototype.

\subsection{High-fi Prototype}
\subsubsection{News Site}
Two High-fi prototypes was made from the online news site bbc \cite{bbc} and QQ \cite{qq_homepage}. These prototypes was directly modelled from the websites and then the corresponding logos was removed. The websites was also both translated to English respectivly Chinese. The High-fi prototypes can be seen in the following figures: QQ (CITE to QQ image), BBC (SITE TO BBC IMAGE)





\section{Discussion Phase 2}
Both of the pages were modelled from a combination of big websites that are already existing. Because of this the low-fi prototype was not necessary to test. The high-fi prototype on the other hand needed to be tested together with the questions to make sure these were understandable. The pilot study was also conducted to make sure that the main test is feasible. 

\section{Conclusion}


