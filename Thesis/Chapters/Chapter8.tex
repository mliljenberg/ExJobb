% Chapter Template

\chapter{Results} % Main chapter title

\label{Chapter8} % Change X to a consecutive number; for referencing this chapter elsewhere, use \ref{ChapterX}

%----------------------------------------------------------------------------------------
%	SECTION 1
%----------------------------------------------------------------------------------------
Each test was only done once per user meaning that all the users only saw the QQ site or the BBC site in their language. The questionnaire contained the same questions no matter which site the user performed their test on.

\section{BBC English}
For the BBC English website, there were a total of 29 participants. Most respondents were in the age ranges of 18-24 and 25-35 with few outliers. All users who saw the site were self-proclaimed English speakers and were asked to navigate the BBC website using English, these users only saw the BBC English site and not the BBC Chinese site. On average, native English speakers using the BBC site, correctly responded to task queries 88\% of the time (see fig \ref{fig:bbc_results}). Additionally, the mean time per question used to answer questions accurately was 21.5 seconds. To complete the tests, users took an average of 5.5 minutes, where the mean time spent per question was 25.3 seconds (see fig \ref{fig:bbc_results}). For questions that fell within the F-shaped pattern, participants used roughly 19.1 seconds to answer a question. Conversely, users took 31.8 seconds to answer questions outside the F-shaped pattern. The difference between how long it took English users to find answers within the F-shaped pattern (19.1 seconds) and outside the F-shaped pattern (31.8 seconds) proved to be statistically significant (t = 2.8479, p < 0.05), indicating that English speakers were quicker when navigating the BBC site within the F-pattern compared to outside. 
\\\\
The Questionnaire results can be seen in \ref{sec:sus_res}. The overall calculated result for the BBC English site based on the Questionnaire is 4.88. Where 0 is the lowest score and 10 is the highest.

\section{BBC Chinese}
For the BBC Chinese website, there were a total of 21. Again, most respondents were in the age ranges of 18-24 and 25-35 with few outliers, indicating a rather homogeneous response group. All users who saw the site were self-proclaimed Chinese speakers and were asked to navigate the BBC website using Chinese, these users only saw the BBC Chinese site and not the BBC English site. On average, native Chinese speakers using the BBC site, correctly responded to task queries 74\% of the time (see fig \ref{fig:bbc_results}). Additionally, the mean time per question used to answer questions accurately was 24 seconds. To complete the tests, users required roughly of 5.9 minutes, where the mean time spent per question was 27.3 seconds (see fig \ref{fig:bbc_results}). For questions that fell within the F-shaped pattern, Chinese participants used roughly 24.4 seconds to answer a question. Conversely, Chinese users took 28.6 seconds to answer questions outside the F-shaped pattern. The difference between how long it took Chinese users to find answers within the F-shaped pattern (24.4 seconds) and outside the F-shaped pattern (28.6 seconds) proved to be statistically significant (t = 5.3301, p < 0.05), indicating that Chinese speakers were, surprisingly, slightly quicker when navigating the BBC site within the F-pattern compared to outside. 
\\\\
The Questionnaire results can be seen in \ref{sec:sus_res}. The overall calculated result for the BBC English site based on the Questionnaire is 5.51. Where 0 is the lowest score and 10 is the highest.

\section{Differences between Chinese and English BBC users
}
When comparing the results of Chinese users and English users for BBC, English users, who answered 88\% of questions accurately, tended to get questions correct more often. In comparison, Chinese users only answered 74\% of questions correctly on the BBC site. The difference in the number of correct responses were found to be not statistically significant at the p=0.05 level of significance (t = 1.8893, p>0.05). Interestingly, there appeared to be no difference between the proportion of questioned answered correctly between the Chinese and English BBC users. Additionally, although Chinese users spent more time on average to respond to questions, taking an average of 5.9 minutes whereas English users only took 5.5 minutes, the two groups were not statistically different in their mean response time rates (t = -0.49, p > 0.05). These results indicate that there is no major difference, beyond random chance, that Chinese users or English differ from each other in terms of time spent on answering questions and the tendency to answer questions correctly. 

\begin{figure}[h]
	\centering
	\includegraphics[width=100mm]{Images/bbc_correct_res.png}
	\includegraphics[width=100mm]{Images/bbc_time_res.png}
	\decoRule
	\caption[BBC Results]{The Correct and Mean time per correct question results from the bbc page.}
	\label{fig:bbc_results}
\end{figure}
\newpage

\section{QQ English }
For the QQ English website, there were a total of 30 participants. All respondents were in the age ranges of 18-24 and 25-35. All users who saw the site were self-proclaimed English speakers and were asked to navigate the QQ website using English, these users only saw the QQ English site and not the QQ Chinese site. On average, native English speakers using the QQ site correctly responded to task queries 87\% of the time (see fig \ref{fig:qq_results}). Additionally, the mean time per question used to answer questions accurately was 17.7 seconds (see fig \ref{fig:qq_results}). To complete the tests, users took an average of 4.3 minutes, where the mean time spent per question was 19.3 seconds. For questions that fell within the F-shaped pattern, participants used roughly 17.6 seconds to answer a question. Conversely, users took 20.7 seconds to answer questions outside the F-shaped pattern. The difference between how long it took English users to find answers within the F-shaped pattern (17.6 seconds) and outside the F-shaped pattern (20.7 seconds) proved to be statistically significant (t = -3.1606, p < 0.05), indicating that English speakers were quicker when navigating the QQ site within the F pattern compared to outside. 
\\\\
The Questionnaire results can be seen in \ref{sec:sus_res}. The overall calculated result for the BBC English site based on the Questionnaire is 2.62. Where 0 is the lowest score and 10 is the highest.

\section{QQ Chinese}
For the QQ Chinese website, there were a total of 21 participants. Again, all respondents were in the age ranges of 18-24 and 25-35, indicating a rather homogenous response group. All users who saw the site were self-proclaimed Chinese speakers and were asked to navigate the QQ website using Chinese, these users only saw the QQ Chinese site and not the QQ English site. On average, native Chinese speakers using the QQ site correctly responded to task queries 85\% of the time (see fig \ref{fig:qq_results}). Additionally, the mean time per question used to answer questions accurately was 18.5 seconds (see fig \ref{fig:qq_results}). To complete the tests, users required roughly of 4.9 minutes, where the mean time spent per question was 22.5 seconds. For questions that fell within the F-shaped pattern, Chinese participants used roughly 22.6 seconds to answer a question. Conversely, Chinese users took 22.4 seconds to answer questions outside the F-shaped pattern. The difference between how long it took Chinese users to find answers within the F-shaped pattern (22.6 seconds) and outside the F-shaped pattern (22.4 seconds) proved to be statistically significant (t = -5.907, p <0.05), indicating that Chinese speakers were, slower when navigating the QQ site within the F pattern compared to outside. 
\\\\
The Questionnaire results can be seen in \ref{sec:sus_res}. The overall calculated result for the BBC English site based on the Questionnaire is 4.46. Where 0 is the lowest score and 10 is the highest.

\section{Differences between Chinese and English QQ users}
When comparing the results of English users and Chinese users for QQ, English users, who answered 87\% of questions accurately, tended to once again get questions correct more often. Again, by comparison, Chinese users, who got 85\% of questions correct, answered fewer questions accurately on the QQ site. The difference in the number of correct responses were not statistically significant at the p=0.05 level of significance (t = 0.42, p>0.05). Interestingly, there appeared to be no difference between the proportion of questioned answered correctly between the Chinese and English QQ users, just as for the BBC users. Additionally, although Chinese users spent more time on average to respond to questions, taking an average of 4.9 minutes whereas English users only took 4.3 minutes, the two groups were not statistically different in their mean response time rates (t = -0.49, p > 0.05), once again, the results are similar to BBC. These findings suggest there is no major difference, beyond random chance, that Chinese users or English differ from each other in terms of time spent on answering questions and the tendency to answer questions correctly.

\begin{figure}[h]
	\centering
	\includegraphics[width=100mm]{Images/qq_correct_res.png}
	\includegraphics[width=100mm]{Images/qq_time_res.png}
	\decoRule
	\caption[QQ Results]{The Correct and Mean time per correct question results from the qq page.}
	\label{fig:qq_results}
\end{figure}

\section{Questionnaire Results} 
\label{sec:sus_res}

\begin{center}
	\begin{tabular}{ | p{5cm} | l | l |  l | l |}
		\hline
		Question & QQ Chinese & QQ English & BBC Chinese & BBC English \\ \hline
		1: I liked the design of the site &
		2,52 &
		1,6 &
		2,95 & 
		2,34  \\ \hline
		2: The design of this site was similar too other news sites &
		3,9 &
		1,97 &
		3,24 &
		3,45  \\ \hline
		3: I think that I would like to use this site frequently &
		2,19 &
		1,33 &
		2,33 &
		1,97  \\ \hline
		4: I thought the site was easy to use &
		2,48 &
		1,6 &
		3,14 &
		2,72  \\ \hline
		5: The design of this site was unusual to me &
		2,71 &
		3,73 &3
		2,52 &
		2,41  \\ \hline
		6: I thought there was too much inconsistency in this site &
		3,1 &
		3,67 &
		2,71 &
		2,89  \\ \hline
		7: I felt very confident using the site &
		2,52 &
		2,1 &
		2,9 &
		2,76 \\ \hline
		8: I thought the material i was looking for was easy to find &
		2,38 &
		1,73 &
		2,9 &
		2,48 \\ \hline
		9: I found the site very cumbersome to use &
		3,48 &
		3,93 &
		3,05 &
		3,07 \\ \hline
		10: I thought that the amount of information on this site was too sparse &
		2,43 &
		2,33 &
		2,9 &
		2,69 \\ \hline
		11: I felt overwhelmed using this site &
		2,81 &
		4,23 &
		2,71 &
		2,76 \\ \hline
		
	\end{tabular}
\end{center}

\begin{figure}[h]
	\centering
	\includegraphics[width=150mm]{Images/sus_res.png}
	\decoRule
	\caption[Questionnaire Results]{The results from the Questionnaire}
	\label{fig:sus_results}
\end{figure}



