% Chapter Template

\chapter{Discussion} % Main chapter title

\label{Chapter9} % Change X to a consecutive number; for referencing this chapter elsewhere, use \ref{ChapterX}

%----------------------------------------------------------------------------------------
%	SECTION 1
%----------------------------------------------------------------------------------------



In this research, I aimed to investigate 1) how differences in interface design may be due to differences in information processing styles or simply trends, 2) how do different processing styles in Western (analytical) versus Chinese (holistic) users affect performance on different interfaces, and 3) if one global interface be created, or if web designers focus on creating separate user interfaces for different cultures. Concerning the question of how interface design may be due to different information processing styles or trends, my research suggests that a mix of both factors have contributed to the different web designs from China and its western counterparts. Specifically, Chinese users preferred the western based design over the Chinese one, according to self-reports [report stats XXX]. Interestingly, this suggests that perhaps the reason why Chinese news sites tend to be more information dense may simply be due to a trend in web design rather than fundamental perception differences. For instance, Chinese users performed as well as English users navigating both the BBC site (t = -0.49, p > 0.05) and QQ site (t = 0.42, p>0.05) indicating that there is no statistical difference between the two groups. 
\\\\
Regarding the question of how information processing styles western (analytical) versus eastern (holistic) would affect user performance on different interfaces, my research suggests that information processing style does significantly influence performance. Unsurprisingly, my research indicated that English, or analytical, users who used the BBC site were much quicker at finding important objects within the F-shaped pattern compared to outside of it (t = 2.8479, p < 0.05), taking an outstanding 12 seconds longer outside the F-shaped pattern. This is congruent with the existing literature on UX, which tends to be western-centric, as eye tracking studies suggest western users tend to scan through a website using an F-shaped pattern. However, it was surprising to find that Chinese users were actually also able to answer questions more quickly when questions laid within the F-shaped pattern versus outside (t = 5.3301, p < 0.05). It appears that for both Chinese and English users of the BBC site, objects that fell within the F-pattern were easier to find that those outside. However, it appears that the effect size was significantly larger for English users, with a 12 second difference inside and outside the F-pattern, than for Chinese users, who only had a 4 second difference. Overall, it appears that a western based design seems easily navigable for both eastern and western users and the f-shape appears to increase performance for both groups. The main difference, however, is that English users were significantly better within the F-pattern (taking 12 seconds less) than outside, whereas Chinese users only saw a slight improvement (taking only 4 seconds less within the F-shape). 
\\\\
On the effects of information processing and website navigability, my findings suggest that English speakers using the QQ site were quicker inside the F-shaped pattern than outside of the pattern (t = -3.1606, p < 0.05), with a 3 second difference. For Chinese users on the QQ site, conversely, they were very marginally slower inside the F-pattern (using 22.6 seconds) than outside (22.4 seconds) the pattern (t= -5.907, p <0.05), suggesting that Chinese users using the QQ site did not rely on using the F-shaped pattern as much as western users. Overall, it appears that even though both Chinese and English users were able to navigate the QQ site with the same speed (t = -0.49, p > 0.05), English users were still slightly more dependent on the F-pattern, like the findings form the BBC site. Interestingly, however, according the questionnaire self-reports, it appeared that Chinese users found both the QQ and BBC websites to be equally likeable 4.46 (QQ) vs 5.51 (BBC) and did not seem to be overwhelmed by the information density of the QQ site, average of 2.8 from 1 to 5 on the question "I felt overwhelmed using this site". For the English users, however, self-reports indicated that they were highly uncomfortable with the layout of the page, finding the information overwhelming. They had a average of 4.23 from 1 to 5 on the question "I felt overwhelmed using this site". They also tended to give higher ratings for BBC than QQ 4.88 (BBC) vs 2.62 (QQ). 
\\\\
Taken together, these results suggest that information processing does influence how users interact with different interfaces. It appears that while western users can use a Chinese designed site, they find the experience to be extremely unpleasant 2.62 (QQ). Conversely, my findings indicate that Chinese users find both sites equally likeable 4.46 (QQ) vs 5.51 (BBC). Furthermore, English users were more reliant on the F-shaped pattern than Chinese users were, which is in-line with existing research on user perception. Concerning the question of whether or not one global interface should be created or if web designs should be tailored to different cultures, my research suggests that one global interface can be deployed to maximize efficiency. Chinese users found BBC to be just as likeable as QQ, whereas English users disliked the information density of QQ, suggesting that a western designed site with sleek information layouts appear to be generally liked by users. Accordingly, it appears that web designers can focus on creating one global interface rather than tailoring websites to different cultures. 
