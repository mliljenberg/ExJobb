% Chapter 1

\chapter{Introduction} % Main chapter title

\label{Introduction} % For referencing the chapter elsewhere, use \ref{Chapter1} 

%----------------------------------------------------------------------------------------

% Define some commands to keep the formatting separated from the content 
% Define some commands to keep the formatting separated from the content 
\newcommand{\keyword}[1]{\textbf{#1}}
\newcommand{\tabhead}[1]{\textbf{#1}}
\newcommand{\code}[1]{\texttt{#1}}
\newcommand{\file}[1]{\texttt{\bfseries#1}}
\newcommand{\option}[1]{\texttt{\itshape#1}}

%----------------------------------------------------------------------------------------


This master thesis has been completed for the Department of Design Sciences at Lund University in collaboration with the company Tetra pak. The first chapter will introduce the background, purpose and scope.

\section{Background}
Working in Shanghai for a summer, I noticed that there was a big difference in how Chinese websites where designed compared to western sites. I found the Chinese sites very overwhelming in information density. I assumed that this was because of a simple difference in cultural trends. Later I came across some cross-cultural research articles proving that there is a difference in how people from western and eastern societies perceive information. As such, I wanted to investigate if this also influences how the Chinese web pages are designed? Together with Tetra pak, this thesis project is created to examine the differences in web design in eastern and western cultures and to find out if a global interface for both cultures can be created.

\section{Global Website design}
Designing websites for a global market is something that is becoming increasingly necessary for large multinational companies to do. In spite of the need and popularity of global designed websites there is a lack of research in this field for user experience. Many companies simply try to launch their local product globally and hope the design works everywhere. Some companies, conversely, simply create a new product for the different market without researching if this is necessary. It is evident that websites look different in cultures all over the world. But how much of this difference is because of cultural trends? Are parts of these websites designed after a difference in how individuals from different cultures perceive information? How well can users from different cultures perform on different web elements?
\\\\
There is a surprising lack of research done on this subject. Despite this, however, companies over the world continue to spend millions attempting to build information products for cultures without relevant knowledge about how users from that culture perceive information. There are a quite a few assumptions that are commonly made about users in different cultures. The first and most common assumption is that all users process information the same way. Although this assumption has been found false, most people believe that the differences in web layout are mostly due to a difference in language or simply trends. The second assumption is that because eastern websites are designed differently, easterners prefer this type of design. There are quite a lot of ambiguities when examining web pages and web apps in different cultures and hopefully, this thesis will resolve some of these unclarities.

\section{Tetra Pak}
Tetra Pak is one of the world's leading food processing and packaging solutions company.  With more than 24,000 employees around the world they are one of the biggest actors when it comes to food safety. Tetra pak provide everything from factory processing, packaging, automated plant solutions to web based systems that helps their users control their factories.

\section{Purpose}
The purpose of this thesis is to research the differences between western and eastern website usage, and how differences in perception plays a role in this. To do this we will try to answer the following questions:
 \begin{enumerate}
	\item Are differences in interface design due to differences in information processing styles or trends?
	\item Do different processing styles in Western (analytical) versus Chinese (holistic) users significantly affect performance on different interfaces?
	\item Can one Global interface be created, or should web designers focus on creating separate user interfaces for different cultures?
\end{enumerate}

This thesis is done in partnership with Tetra pak since they currently have development ongoing in many eastern countries, such as China. Tetra pak is interested in trying to learn more about user-experience in these eastern countries to help them improve their products in these countries. Using Tetra pak resources from China allowed this thesis to be conducted and is a reason for limiting the testing to Chinese and Swedish users. 

\section{Project scope}
The duration of this project is limited to 20 weeks that, once completed, will deliver answers to the questions defined in the purpose above. This time limit does create some restrictions for the study.
\begin{enumerate}
	\item Majority of time had to be spent on the development phase since this is the most crucial part to be able to gather any data at all.
	\item Once a certain time had past no more data could be gathered from users. This for the thesis to be finished in time. Even if this could possibly lead to more statistically sound results.
	\item Focus had to be put on a select few user behaviours. Once again because of the time limitations.
\end{enumerate}


