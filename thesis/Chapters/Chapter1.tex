% Chapter 1

\chapter{Introduction} % Main chapter title

\label{Introduction} % For referencing the chapter elsewhere, use \ref{Chapter1} 

%----------------------------------------------------------------------------------------

% Define some commands to keep the formatting separated from the content 
% Define some commands to keep the formatting separated from the content 
\newcommand{\keyword}[1]{\textbf{#1}}
\newcommand{\tabhead}[1]{\textbf{#1}}
\newcommand{\code}[1]{\texttt{#1}}
\newcommand{\file}[1]{\texttt{\bfseries#1}}
\newcommand{\option}[1]{\texttt{\itshape#1}}

%----------------------------------------------------------------------------------------


This master thesis has been completed for the Department of Design Sciences at Lund University in collaboration with the company Tetra pak. This first chapter will introduce the background, purpose and scope.

\section{Background}
Working in Shanghai for a summer I noticed that there was a big difference in how Chinese websites where designed compared to western sites. I found the Chinese sites very overwhelming in information density. I assumed that this was because of a simple difference in cultural trends. Later I came across some cross-cultural research articles proving that there is a difference in how people in western and eastern societies perceive information. Does this also influence how the Chinese web pages are designed? Together with Tetra pak, this thesis project is created to examine the differences in web design in eastern and western cultures and to find out if a global interface for both cultures can be created.

\section{Global Website design}
Designing websites for a global market are something that is becoming more and more necessary for larger companies to do. In spite of the need and popularity of global designed websites not much research has been made in this field for user experience. Many companies simply try to launch their local product globally and hope the design work everywhere. Some companies simply create a new product for the different market without researching if this is necessary or not. It is clear that websites look different in cultures all over the world. But how much of this difference is because of cultural trends? Are parts of these websites been designed after a difference in how the culture perceive information? How well can do users from different cultures perform on different web elements?

There is a surprising lack of research done on this subject, despite this, companies all over the world spend millions trying to build information products for cultures without the least bit of knowledge about how the users from that culture perceive information. There are a quite a few assumptions that are usually made about users in different cultures. The first and most common assumption is that all users process information the same way. This is proven to not be the case despite this most people believe that the differences in web layout are mostly due to a difference in language or due to a trend. The second assumption is that because eastern websites are designed differently the users there prefer this type of design, or this type of design is more suitable for users of that culture since it is what they are used to. There are quite a lot of unclarities when examining web pages and web apps in different cultures and hopefully, this thesis will resolve some of these unclarities.

\section{Tetra Pak}

\section{Limitations}

\section{Purpose}
The purpose of this thesis is to research the differences between western and eastern website usage, and how differences in perception plays a role in this. To do this we will try to answer the following questions:
 \begin{enumerate}
	\item Are differences in interface design due to differences in information processing styles or trends?
	\item Do different processing styles in Western (analytical) versus Chinese (holistic) users significantly affect performance on different interfaces?
	\item Can one Global interface be created, or should web designers focus on creating separate user interfaces for different cultures?
\end{enumerate}

Tetra pak is a global company that is 
		

\section{scope}
Focus on diffrences between china and sweden


